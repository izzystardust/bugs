\documentclass[12pt]{article}

\title{Mobile Robotics and HRI: Homework 2}
\author{Ethan Miller\\0278718}
\date{\today}

\usepackage[margin=1in]{geometry}

\begin{document}
\maketitle

The robot state space consisted of two components: a map of functions, keeping
track of how the robot was accomplishing its current goal, and a set of booleans,
keeping track of how the robot was changing its goals and prompting the user.

The core of the robot execution is the Bug.step() method, which is called once
every 10ms. It first updates the state based on the map of state functions, which
each return the next state that should be entered if no other factors applied. It
then checks each of the conditions for prompting the user and the booleans
controlling their triggers, asking and changing the goal and speed as appropriate.

The only exceptions to this flow of control are two timers: One that decrements
the battery, and another that triggers after two minutes to tell the robot to
return to the goal. It is worth noting that while the robot waits for user input,
this two minute timer continues to count down and may cause the robot to act
differently than expected.

To prompt the user, the robot asks a binary question; that is, a question with
only two possible answers. One is indicated as the default choice. If any answer
besides the nondefault choice is made, the default choice is taken instead of
reprompting the user. Because the default answers are ``safe'' answers, this
approach was taken over reprompting until a valid answer is given; additionally,
this approach is easier programatically. While waiting for user answers, the
robot pauses all actions.

Alternative prompts were considered, but discarded. A graphical prompt was
left behind due to relative difficulty for a relatively insignificant gain in
obviousness given the safety of the actions taken by the simulated robot.
Contacting the user via phone fell out of the running for the same. The prompts
could also be on the robot, but a simulated square robot made that impractical.

To inform the user of decisions made, the robot prints to the console. Due to the
safety and low-priority of the user acknowledging these decisions, this approach
is defensible. If the user's awareness was a priority, alternative methods of
informing would be necessary: options include cell phone notifications, active
prompts on the computer (either at the command line or as a GUI), or an indication
of such on the physical robot.

\end{document}
